\documentclass[nonanonymous]{article}

% if you need to pass options to natbib, use, e.g.:
%     \PassOptionsToPackage{numbers, compress}{natbib}
% before loading neurips_2020

% ready for submission
% \usepackage{neurips_2020}

% to compile a preprint version, e.g., for submission to arXiv, add add the
% [preprint] option:
%     \usepackage[preprint]{neurips_2020}

% to compile a camera-ready version, add the [final] option, e.g.:
%     \usepackage[final]{neurips_2020}

% to avoid loading the natbib package, add option nonatbib:


\usepackage[utf8]{inputenc} % allow utf-8 input
\usepackage[T1]{fontenc}    % use 8-bit T1 fonts
\usepackage{hyperref}       % hyperlinks
\usepackage{url}            % simple URL typesetting
\usepackage{booktabs}       % professional-quality tables
\usepackage{amsfonts}       % blackboard math symbols
\usepackage{nicefrac}       % compact symbols for 1/2, etc.
\usepackage{microtype}      % microtypography

\title{Project Proposal for CS7150}

% The \author macro works with any number of authors. There are two commands
% used to separate the names and addresses of multiple authors: \And and \AND.
%
% Using \And between authors leaves it to LaTeX to determine where to break the
% lines. Using \AND forces a line break at that point. So, if LaTeX puts 3 of 4
% authors names on the first line, and the last on the second line, try using
% \AND instead of \And before the third author name.

\author{
  Matthew Kazan \\
  Khoury College of Computer Science\\
  Northeastern University\\
  Boston, MA 02115 \\
  \texttt{kazan.m@northeastern.edu} \\
}

\begin{document}

\maketitle

\section{Overview of Project Idea}

Simultaneous Localization and Mapping (SLAM) is a fundamental problem in robotics,
particularly for autonomous navigation and 3D reconstruction.
This project aims to enhance a LiDAR-based SLAM pipeline using deep learning
for three critical components: loop closure detection, outlier rejection,
and pose estimation.
Additionally, a end-to-end deep learning SLAM model will be implemented to compare against
the hybrid approach.

The foundation of this project builds on prior work that involved developing a ROS2 pipeline
and iPhone LiDAR scanning setup aiming to construct a global map of the environment.
However, this previous work was strictly background software development—focused
on data collection and system setup—without incorporating any deep learning methods.
Since the research contribution of this project lies in the novel application of
deep learning techniques to SLAM, it represents a distinct extension of prior efforts.

Key Contributions:

\begin{enumerate}
\item Loop Closure Detection: Implement a deep learning-based method
to detect revisited locations, reducing accumulated drift.
\item Outlier Rejection: Train a neural network to filter out poor correspondences
before ICP refinement, improving alignment robustness.
\item Pose Estimation: Use a deep learning model (e.g., Deep Closest Point (DCP))
to estimate a transformation between scans to construct a global map.
We can also compare this to the ICP or other traditional point cloud registration algorithms.
\item End-to-End Deep Learning SLAM: Implement a learning-based SLAM model that estimates poses and
maps directly from LiDAR data, serving as a comparison point against the ICP-based hybrid approach.
\end{enumerate}
Performance will be evaluated based on localization accuracy, map consistency, and computational efficiency.


\subsection*{Literature Survey}
This project builds on existing research in deep learning-based SLAM and point cloud registration.
Key related papers include:

\begin{enumerate}
\item G. Kim et al., "Scan Context: Egocentric Spatial Descriptor for Place Recognition Within 3D Point Cloud Map," IEEE T-RO 2018.

\begin{itemize}
\item Introduces a compact representation for place recognition in 3D point clouds, enabling efficient loop closure detection.
The method encodes spatial relationships between keypoints, making it highly suitable for real-world SLAM applications.


\end{itemize}

\item S. Rakotosaona et al., "PointCleanNet: Learning to Denoise and Remove Outliers from Dense Point Clouds," CVPR 2020.

\begin{itemize}
\item Proposes a learning-based approach for denoising point clouds and removing outliers.
The model is trained to differentiate between noise and meaningful structure, ensuring more reliable SLAM mapping, particularly in cluttered or dynamic environments.


\end{itemize}

\item H. Li et al., "DeepICP: An End-to-End Deep Neural Network for 3D Point Cloud Registration," CVPR 2019.

\begin{itemize}
\item Explores a deep learning alternative to ICP for point cloud registration, demonstrating improved robustness to initialization errors and noise.
This work could be used as the end-to-end deep learning SLAM model, as it directly estimates the transformation between scans.

\end{itemize}

\item C. Choy et al., "Deep Global Registration," CVPR 2020.

\begin{itemize}
\item Introduces a global feature-based registration method using deep learning.
This approach allows for more reliable registration in challenging environments and is particularly useful for end-to-end deep learning SLAM systems.
The method avoids local minima problems common in traditional registration algorithms, ensuring more stable map construction.

\end{itemize}

\end{enumerate}



\subsection*{Potential Datasets}
\begin{enumerate}
\item KITTI Odometry Dataset – Provides LiDAR scans and ground truth poses for benchmarking SLAM accuracy.


\item ETH LiDAR Benchmark Dataset – Contains real-world LiDAR scans for evaluating outlier rejection and loop closure detection.


\item Custom iPhone LiDAR Data – Collected using a ROS2-integrated iPhone LiDAR scanner to validate performance on real-world mobile mapping tasks.


\end{enumerate}

\subsection*{Plan of Activities}
\begin{enumerate}
\item Week 1:Set up ROS2 pipeline, process iPhone LiDAR scans into structured point cloud datasets.

\item \item Week 2: Implement algorithms for point cloud registration using ICP and other deep learning methods.

\item Week 2--3: Implement deep learning models for loop closure detection using Scan Context and outlier rejection using PointCleanNet.

\item Week 4--5:Develop a fully deep learning SLAM system, incorporating Deep Global Registration to estimate pose and build a global map.

\item Week 6:Conduct experiments, compare deep learning SLAM vs.\ ICP baseline, finalize project report.


\end{enumerate}
This plan ensures a deep learning-based SLAM approach is developed, tested, and benchmarked before submission.



\end{document}