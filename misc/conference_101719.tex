\documentclass[conference]{IEEEtran}

\usepackage{graphicx}
\usepackage{amsmath,amssymb,amsfonts}
\usepackage{algorithmic}
\usepackage{textcomp}
\usepackage{xcolor}
\usepackage{cite}

\begin{document}

    \title{Indoor SLAM Using ICP and an iPhone}

    \author{
        \IEEEauthorblockN{Matthew Kazan}
        \IEEEauthorblockA{\textit{Khoury College of Computer Sciences} \\
        \textit{Northeastern University} \\
        Boston, MA \\
        kazan.m@northeastern.edu}
    }

    \maketitle

    \begin{abstract}
        This paper presents the development of a system for 3D mapping using Iterative Closest Point
        (ICP) Simultaneous Localization and Mapping (SLAM). Leveraging the LiDAR scanner on an
        iPhone, the system collects sequential point cloud data, which is transmitted to a web
        server for processing.
        The server applies ICP to align the scans and generate a global 3D map of the environment.
        This approach demonstrates the feasibility of constructing accurate, low-cost 3D maps
        using common, readily available hardware.
        Results showcase successful alignment of point clouds and the generation of a cohesive room-scale map.
        Challenges such as managing noise and drift are also discussed, alongside potential improvements.
    \end{abstract}



    \section{Introduction}
    Accurate 3D mapping has many applications in robotics, indoor navigation, and augmented reality.
    Traditional mapping systems often require expensive, specialized hardware, which limits accessibility.
    The LiDAR scanner embedded in consumer-grade devices, such as the iPhone, presents a unique opportunity
    to apply Simultaneous Localization and Mapping (SLAM) techniques to real-world data without requiring
    specialized equipment.

    Implementing SLAM using an iPhone's LiDAR scanner provides a practical opportunity to deepen
    understanding of ICP-based SLAM techniques.
    This project bridges theoretical concepts from lectures with hands-on applications,
    enhancing both learning and technical skill development.

    \section{Problem Statement}
    The problem studied in this project involves constructing a global
    3D map of an indoor environment by aligning sequential LiDAR scans.
    Mathematically, the task is to solve for the transformation matrix $T = [R|t]$, where $R$
    is a rotation matrix, and $t$ is a translation vector that aligns consecutive point clouds
    $X$ and $Y$: \begin{equation} T = \arg\min_T \sum_{i=1}^N |y_i - T x_i|^2, \end{equation}
    where $x_i \in X$ and $y_i \in Y$ are corresponding points in the two scans.
    Challenges include handling noise, managing drift over multiple scans, and ensuring computational efficiency.

    \section{Proposed Solution}
    The proposed method leverages the Iterative Closest Point (ICP) algorithm to align consecutive
    point clouds collected by the iPhone's LiDAR scanner.
    The pipeline includes:

    \subsection{Data Collection}
    A custom iOS app captures LiDAR scans at 0.1-second intervals and uploads them to a local web server.

    \begin{figure}[htbp]
        \centering
        \includegraphics[width=0.28\textwidth]{iphone_app}
        \caption{Screenshot of the iPhone app used to upload scans to the web server.}
        \label{fig:iphone_app}
    \end{figure}

    \subsection{Web Server}
    A Flask server enables users to upload scans to a PostgreSQL database and clear the database as needed.

    \subsection{Point Cloud Processing}
    A continuous loop checks the database for new scans and processes them.
    The scans are converted from pixel coordinates to 3D coordinates using the known camera intrinsics of the iPhone.

    \subsection{ICP Alignment}
    ICP is applied to align the new scan with the global map using the Open3D point-to-point
    ICP transformation estimation function.
    The aligned scan is added to the global map, which is placed in a thread-safe queue for the GUI to display.
    Occasionally, statistical outliers in the global map are removed, and the result is downsampled
    to reduce complexity and memory usage.

    \subsection{Point Cloud GUI}
    An interactive Open3D GUI displays the global point cloud map as it is updated.
    The map can be inspected, zoomed, and rotated in real time.
    Once the processing thread completes, the global map is saved to a .ply file for further analysis.

    \begin{figure}[htbp]
        \centering
        \includegraphics[width=0.48\textwidth]{desk_in_map}
        \caption{Screenshot of a desk and chair displayed on the GUI as a point cloud.}
        \label{fig:desk_in_map}
    \end{figure}

    \subsection{Point-to-Point Algorithm}
    The point-to-point ICP algorithm minimizes the Euclidean distance between corresponding points
    in the source point cloud $P$ and the target point cloud $Q$.
    Mathematically, the cost function is:
    \begin{equation}
        C(T) = \sum_{i} ||y_i - (R_k x_i + t_k)||^2,
    \end{equation}
    where:
    \begin{itemize}
        \item $T_k = [R_k|t_k]$ is the current transformation matrix estimate with rotation $R$ and translation $t$.
        \item $x_i \in X$ is a point in the source cloud.
        \item $y_i \in Y$ is the corresponding point in the target cloud.
    \end{itemize}
    The algorithm iteratively updates $R$ and $t$ until convergence criteria, such as a threshold
    on the cost function change or a maximum number of iterations, are met.

    \section{Results}
    The performance of the proposed system was evaluated using the following metrics:
    \begin{itemize}
        \item \textbf{Mean Squared Error (MSE):} Average squared distance between corresponding points in consecutive aligned scans.
        \item \textbf{Hausdorff Distance:} Maximum distance from a point in one point cloud to the nearest point in the other point cloud.
        \item \textbf{Mean Distance:} Average distance between corresponding points in the aligned scans.
    \end{itemize}

    Experimental results demonstrated that the system achieved an MSE of 1.23 cm$^2$
    , a Hausdorff distance of 0.8 meters due to outliers, and a mean distance of 4.58 cm.
    Given that the room size is approximately 3 meters by 3 meters, the MSE and mean distance error
    represents around 1\% of the room's dimensions, indicating high precision in alignment.
    These metrics validate the system's effectiveness in producing accurate 3D maps.
    However, drift was observed, with the last and first scans misaligned by approximately 20 cm
    due to cumulative errors.

    \begin{figure}[htbp]
        \centering
        \includegraphics[width=0.40\textwidth]{overlay_with_gt}
        \caption{Screenshot of the global map overlaid with the ground truth for comparison.}
        \label{fig:overlay_screenshot}
    \end{figure}

    \section{Conclusion}
    This project demonstrates the feasibility of constructing 3D maps using ICP SLAM with an iPhone's LiDAR scanner.
    The system achieves reasonable accuracy and efficiency, highlighting the potential of
    low-cost hardware for 3D mapping.

    However, several challenges were encountered during the development and evaluation process.
    Drift over time emerged as a significant issue, particularly in longer sequences,
    where cumulative alignment errors caused noticeable deviations.
    Rapid camera movements also led to poor alignment results, as the ICP algorithm struggles
    with large transformations between scans.
    Another challenge was managing outliers in the point clouds, as some remained even after
    applying statistical removal techniques.
    These issues highlight the need for enhancements to improve robustness and accuracy.

    Future work will focus on implementing drift correction mechanisms, such as loop closure
    techniques, to minimize cumulative errors.
    Enhancing outlier removal methods, possibly through more sophisticated filtering algorithms or
    machine learning-based approaches, is also a priority.
    Additionally, optimizing performance to handle larger global maps and faster processing times
    will be crucial for real-time applications.
    Finally, integrating the system into the iPhone app for seamless on-device processing and
    visualization would greatly improve user experience and usability.

    \begin{thebibliography}{00}
        \bibitem{b1} D. Holz and S. Behnke, "Sancta simplicitas - on the efficiency and achievable results of SLAM using ICP-based incremental registration," 2010 IEEE International Conference on Robotics and Automation, Anchorage, AK, USA, 2010, pp. 1380-1387, doi: 10.1109/ROBOT.2010.5509918.
        \bibitem{b2} Surmann, Hartmut & N"uchter, Andreas & Hertzberg, Joachim. (2003). An autonomous mobile robot with a 3D laser range finder for 3D exploration and digitalization of indoor environments. Robotics and Autonomous Systems. 45. 181-198. 10.1016/j.robot.2003.09.004.
        \bibitem{b3} Y. Chen and G. Medioni, "Object modeling by registration of multiple range images," Proceedings. 1991 IEEE International Conference on Robotics and Automation, Sacramento, CA, USA, 1991, pp. 2724-2729 vol.3, doi: 10.1109/ROBOT.1991.132043.
    \end{thebibliography}

\end{document}

